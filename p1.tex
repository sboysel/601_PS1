%%%%% Beginning of preamble %%%%%

\documentclass[12pt]{article}  %What kind of document (article) and what size

%Packages to load which give you useful commands
\usepackage{graphicx}
\usepackage{amssymb, amsmath, amsthm}
\usepackage{amsfonts}
\usepackage{enumitem}
\usepackage{mathtools}
\DeclareMathOperator*{\argmin}{argmin}

%Sets the margins

\textwidth = 7 in
\textheight = 9.5 in
\oddsidemargin = -0.3 in
\evensidemargin = -0.3 in
\topmargin = -0.4 in
\headheight = 0.0 in
\headsep = 0.0 in
\parskip = 0.2in
\parindent = 0.0in

%defines a few theorem-type environments
\newtheorem{theorem}{Theorem}
\newtheorem{corollary}[theorem]{Corollary}
\newtheorem{definition}{Definition}

%%%%% End of preamble %%%%%

\begin{document}

{\Large Sam Boysel} \hfill
{\large Problem 1 - ECON 603}  %Delete one
\hfill  \today

\begin{enumerate}
	\item First, assume \(\epsilon_{i} \sim \text{iid}(\theta_{i}) > 0, \, 
		\forall \, 
		\theta_{i}\) in the parameter space and \(i \in 1, 2\).  Second,
		denote the income stream to agent \(i\) at time \(1\) as
		\(\nu^{i}_{s}\) where \(s \in \{\text{high}, \text{low}\}\) and
		\begin{align*}
			\nu^{i}_{\text{low}} &= b_{i} - \epsilon_i \\
			\nu^{i}_{\text{high}} &= b_{i} + \epsilon_i
		\end{align*}
		and hence \(\nu^{i}_{\text{low}} < \nu^{i}_{\text{high}}\) for
		\(i \in 1, 2\).  Finally, note that the state space at \
		(t = 1\) for agents 1 and 2 can be given by
		\begin{align*}
		S &= \{(\nu^{1}_{s}, \nu^{2}_{s})\}_{s \in \{\text{high},
		\text{low}\}} \\
		&= \{(\nu^{1}_{\text{low}}, \nu^{2}_{\text{low}}),
		     (\nu^{1}_{\text{high}}, \nu^{2}_{\text{high}}),
	     	     (\nu^{1}_{\text{low}}, \nu^{2}_{\text{high}}),
	     	     (\nu^{1}_{\text{high}}, \nu^{2}_{\text{low}})\} \\
		     &= \{\nu^{1}_{\text{high}}, \nu^{1}_{\text{low}}\} \times 
	     		\{\nu^{2}_{\text{high}}, \nu^{2}_{\text{low}}\}	\\
		&= S_1 \times S_2
		\end{align*}
	\item 
\end{enumerate}

\end{document}
